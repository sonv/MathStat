\documentclass[aspectratio=169,handout]{beamer}
\usetheme{Copenhagen}
%% Remove draft for real article, put twocolumn for two columns
\usetheme{metropolis}
\usepackage{multicol}

\usepackage[utf8]{inputenc}
\newtheorem*{question}{Question}

\newcommand{\vectorproj}[2][]{\mathrm{proj}_{\vect{#1}}\vect{#2}}
\newcommand{\vectorcomp}[2][]{\mathrm{comp}_{\vect{#1}}\vect{#2}}
\newcommand{\vect}{\mathbf}
\newcommand{\R}{\mathbb{R}}
%% commentary bubble
\newcommand{\SV}[2][]{\sidenote[colback=green!10]{\textbf{SV\xspace #1:} #2}}

%% Title 
\title{ Multivariable Calculus \\ Day 9\\ Partial derivatives} 
\institute{Fulbright University Vietnam}
%\author[1]{Co-author}
\author{Truong-Son Van}
\date{Spring 2023}

\begin{document}

\maketitle

\section{A word about continuity}
\begin{frame}
    A function $f$ is continuous at $\vect{a}$ if
    \begin{equation*}
        \lim_{\vect{x} \to \vect{a}} f(\vect{x}) = f(\vect{a}) \,.
    \end{equation*}

    $\epsilon -\delta$ definition?
\end{frame}

\begin{frame}
    \frametitle{Partial derivative}
    \begin{equation*}
        f(x,y) = \frac{x^2 \sin (2y)}{32}
    \end{equation*}
\end{frame}


\begin{frame}
    \frametitle{Partial derivative and graph}
    \url{https://www.youtube.com/watch?v=dfvnCHqzK54}
\end{frame}

\begin{frame}
    \frametitle{Formal definition}
Given a function \(f(x,y)\). The partial derivative of \(f\) with respect to \(x\) and \((a,b)\),
denoted by \(f_x(a,b)\), is defined to be
\begin{equation*}
    f_x(a,b) = \lim_{h\to 0} \frac{ f(a+h,b) - f(a,b)}{h} \,.
\end{equation*}
Likewise, the partial derivative of \(f\) with respect to \(y\) and \((a,b)\),
denoted by \(f_y(a,b)\), is defined to be
\begin{equation*}
    f_y(a,b) = \lim_{h\to 0} \frac{ f(a,b+h) - f(a,b)}{h} \,.
\end{equation*}
\end{frame}

\begin{frame}
    \frametitle{Mixed partial derivatives}
    One can keep taking partial derivatives derivatives of higher order
    as each partial derivative again results in a multivariable scalar function.

    \begin{equation*}
        f_{xy} = (f_x)_y = \partial_y(\partial_x f) = \partial_{yx}f \,.
    \end{equation*}

    Be careful with the notations.
\end{frame}

\begin{frame}
    \frametitle{Worksheet}
    Compute all of the mixed derivatives of second order for the functions
    \begin{equation*}
        f(x,y) = e^x \sin y \,,
    \end{equation*}
    and
    \begin{equation*}
        u(x,t) = \sin(x-at) \,,
    \end{equation*}
    where $a$ is a constant.

    What do you notice about the mixed derivatives?
\end{frame}

\begin{frame}
    \frametitle{Worksheet}
    You have seen the normal distribution at least once in your FUV life.

    \begin{enumerate}
        \item Look up the formula for the 2 dimensional normal distribution with mean zero and variance $t$.
        \item Compute all the mixed partial derivatives in terms of $x,y$ (or $x_1, x_2$).
        \item Compute the derivative in $t$ of the normal distribution.
        \item What do you see?
    \end{enumerate}
\end{frame}

\end{document}

