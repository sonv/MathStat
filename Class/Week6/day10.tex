\documentclass[aspectratio=169,handout]{beamer}
\usetheme{Copenhagen}
%% Remove draft for real article, put twocolumn for two columns
\usetheme{metropolis}
\usepackage{multicol}

\usepackage[utf8]{inputenc}
\newtheorem*{question}{Question}

\newcommand{\vectorproj}[2][]{\mathrm{proj}_{\vect{#1}}\vect{#2}}
\newcommand{\vectorcomp}[2][]{\mathrm{comp}_{\vect{#1}}\vect{#2}}
\newcommand{\vect}{\mathbf}
\newcommand{\R}{\mathbb{R}}
%% commentary bubble
\newcommand{\SV}[2][]{\sidenote[colback=green!10]{\textbf{SV\xspace #1:} #2}}

%% Title 
\title{ Multivariable Calculus \\ Day 10\\ Partial derivatives} 
\institute{Fulbright University Vietnam}
%\author[1]{Co-author}
\author{Truong-Son Van}
\date{Spring 2023}

\begin{document}

\maketitle


\begin{frame}
    \frametitle{Last time}
     
    Let
    $f: \mathbb{R}^n \to \R$.

\begin{equation*}
    \partial_{x_i} f(x_1, \dots, x_n) 
    = \lim_{h\to 0} \frac{ f(x_1, \dots, x_{i-1}, x_i + h , x_{i+1}, \dots, x_n) - f(x_1, \dots, x_{i-1}, x_i  , x_{i+1}, \dots, x_n)}{h} 
\end{equation*}

\begin{equation*}
    \partial^m_{x_{k_1}, \dots, x_{k_m}} f = \partial_{x_{k_1}} (\dots ( \partial_{x_{k_m}} f) \dots)
\end{equation*}
where $k_i \in \{1,\dots, n\}$.
\end{frame}



\begin{frame}
\begin{theorem}[Clairaut's Theorem]
Suppose \(f\) is defined on a disk \(D\) that contains the point \((a,b)\).
If the functions \(f_{xy}\) and \(f_{yx}\) are both continuous on \(D\), then
\begin{equation*}
    f_{xy}(a,b) = f_{yx}(a,b) \,.
\end{equation*}
\end{theorem}
\end{frame}


\begin{frame}
    \frametitle{Important notations}

Let \(f:D \to \mathbb{R}\) be a function. We write the following, if exist,
\begin{equation*}
    \nabla f = \begin{bmatrix}
        \partial_{x_1} f\\
        \vdots \\
        \partial_{x_n} f\\
    \end{bmatrix}
\end{equation*}

\begin{equation*}
    \Delta f = \partial_{x_1}^2 f + \dots \partial_{x_n}^2 f \,.
\end{equation*}
\end{frame}


\begin{frame}
    \frametitle{Worksheet}
    Compute directional derivative of the functions
    \begin{enumerate}
        \item $f(x) = |x|, x\in \mathbb{R}^n$
        \item $$ f(x,y) = 
            \begin{cases}
                \frac{x^2y}{x^4 + y^2} & (x,y) \not= 0 \\
                0 & (x,y) = 0 \,.
            \end{cases}$$
        \item $$ f(x) = 
            \begin{cases}
                |x|^2 \sin(1/|x|) & x\not= 0\\
                0 & x = 0 \,.
            \end{cases}
             $$
    \end{enumerate}
\end{frame}


\begin{frame}
    \frametitle{Differentiability}
\begin{definition}
Let \(f:D \to \mathbb{R}\) and \(a\in \mathbb{R}^n\).
Let \(z = f(x)\) and \(\Delta z = f(a + \Delta x ) - f(a)\).
Then \(f\) is \textbf{differentiable at \(a\)} if \(\Delta z\) can be
expressed in the form
\begin{equation*}
    \Delta z = \sum_{i=1}^n \partial_i f(a) \Delta x_i +  \epsilon_i \Delta x_i  \,,
\end{equation*}
where \(\epsilon_i \to 0\) as \(\Delta x_i \to (0,0)\).

\(f\) is said to be \textbf{differentiable} if it is differentiable at every point on the domain.
\end{definition}
\end{frame}


\begin{frame}
    \frametitle{Chain rule}
\begin{theorem}
Let \(f(x_1,\dots, x_n), g_i(y_1,\dots, y_m)\) (\(i = 1,\dots, n\)) be
differentiable
functions.
Then,
\[z(y_1, \dots, y_m) = f(g_1(y_1, \dots, y_m), \dots, g_n(y_1, \dots, y_m))\]
is differentiable and
\begin{equation*}
    \frac{\partial z}{\partial y_i} = \sum_{j=1}^n \frac{\partial f}{\partial x_j} \frac{\partial g_j}{\partial y_i} \,.
\end{equation*}
\end{theorem}
\end{frame}


\begin{frame}
    \frametitle{ Directional derivative  }
\begin{definition}
Let \(\mathbf{u} \in \mathbb{R}^n\). The directional derivative of \(f:\mathbb{R}^n \to \mathbb{R}\) at \(a\in \mathbb{R}^n\)
in the direction of \(\mathbf{u}\) is the following limit (if exists)
\begin{equation*}
    D_{\mathbf{u}} f(a) = \lim_{h \to 0} \frac{ f( a + h \mathbf{u}) - f(a)}{h}\,.
\end{equation*}
\end{definition}
\end{frame}

\begin{frame}
    \frametitle{ Worksheet }
    Compute $\partial_{xy} f$ and $\partial_{yx} f$ of the function
    \begin{equation*}
        f(x,y) = 
        \begin{cases}
            \frac{xy(y^2 - x^2)}{x^2 + y^2}  & (x,y) \not= 0 \,, \\
            0 & (x,y) = 0 \,.
        \end{cases}
    \end{equation*}
\end{frame}




\end{document}

