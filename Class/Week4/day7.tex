\documentclass[aspectratio=169,handout]{beamer}
\usetheme{Copenhagen}
%% Remove draft for real article, put twocolumn for two columns
\usetheme{metropolis}

\usepackage[utf8]{inputenc}
\newtheorem*{question}{Question}

\newcommand{\vectorproj}[2][]{\mathrm{proj}_{\vect{#1}}\vect{#2}}
\newcommand{\vectorcomp}[2][]{\mathrm{comp}_{\vect{#1}}\vect{#2}}
\newcommand{\vect}{\mathbf}
\newcommand{\R}{\mathbb{R}}
%% commentary bubble
\newcommand{\SV}[2][]{\sidenote[colback=green!10]{\textbf{SV\xspace #1:} #2}}

%% Title 
\title{ Multivariable Calculus \\ Day 7 \\ Paramtrization by arc length and introduction multivariable scalar functions} 
\institute{Fulbright University Vietnam}
%\author[1]{Co-author}
\author{Truong-Son Van}
\date{Spring 2023}

\begin{document}

\maketitle

\section{Parametrize by arc length}
\begin{frame}
    \frametitle{Review}

    Arc length =?
\end{frame}

\begin{frame}
    \frametitle{Arc length function}

If one wants to keep track the length of the curve  $\vect{r}:[a,b] \to \R^n$ made by an airplane
at any time $t$, one uses the \emph{arc length function}
\begin{equation*}
    \ell(t) = \int_a^t \left| \vect{r}'(u) \right| \, du \,.
\end{equation*}
\pause
$\ell(t)$ is that it is a strictly increasing function with respect to $t$,
given that $\vect{r}'$ is non-zero for all $t$.
\end{frame}

\begin{frame}
    \frametitle{Worksheet}
    Let $\vect{r}:[-\ln 4,0] \to \R^2$ be a space curve such that
    \begin{equation*}
        \vect{r}(t) = e^t \cos(t) \vect{i} + e^t \sin(t) \vect{j} + e^t \vect{k} \,.
    \end{equation*}
    \begin{enumerate}
        \item Compute the arc length function of this curve.
        \item Is there an inverse function of this?
    \end{enumerate}
\end{frame}

\begin{frame}
    \frametitle{ Re-parametrize with respect to arc length}
    Letting $s = \ell(t)$, we can talk about the inverse of $\ell$, $\ell^{-1}:[0,L] \to [a,b]$
    \begin{equation*}
        t = \ell^{-1}(s) \,.
    \end{equation*}
    \pause
    Therefore, we can re-write
    \begin{equation*}
    \vect{r}(t) = \vect{r}(\ell^{-1}(s)) \,.
    \end{equation*}
\end{frame}

\begin{frame}
    \begin{theorem}
        $$\left| \frac{d \vect{r}(t)}{ds} \right| = 1 \,.$$
Thus,
$$\int_0^s \left| \frac{d}{ds} \vect{r}(t) \right| \, dt = s \,.$$
    \end{theorem}
\end{frame}

\begin{frame}
    \frametitle{Worksheet}
    Verify the theorem with the space curve in the previous problem.
    In particular, show that
        $$\left| \frac{d \vect{r}(t)}{ds} \right| = 1 $$
        and
$$\int_0^s \left| \frac{d}{ds} \vect{r}(t) \right| \, dt = s \,,$$
where
    \begin{equation*}
        \vect{r}(t) = e^t \cos(t) \vect{i} + e^t \sin(t) \vect{j} + e^t \vect{k} \,.
    \end{equation*}
\end{frame}


\begin{frame}
    \begin{definition}
    Let $\vect{T}(t)$ be the unit tangent vector of the curve $\vect{r}:[a,b] \to \R^3$.
    The curvature of $\vect{r}(t(s))$ is defined to be
    \begin{equation*}
        \kappa(s) = \left| \frac{ d \vect{T}(t(s))}{ds} \right| \,.
    \end{equation*}
    \end{definition}

    \pause

    \begin{theorem}
\begin{equation*}
    \kappa(s(t)) =  \frac{|\vect{T}'(t)|}{|\vect{r}'(t)|}   \,.
\end{equation*}
    \end{theorem}
    
    \pause

    Remember the osculating circle?
\end{frame}

\section{Multivariable scalar functions}

\begin{frame}
    \begin{definition}
    Suppose $D$ is a set of $n$-tuples of real numbers $(x_1, x_2, \ldots, x_n)$. 
    A real-valued/scalar function $f$ on $D$ is a rule that assigns a unique (single) real 
    number 
    $$w = f(x_1, x_2, \ldots, x_n)$$ 
    to each element in $D$. 
    The set $D$ is the function's domain. The set of $w$-values taken on by $f$ is 
    the function's range. The symbol $w$ is the dependent variable of $f$, and $f$ 
    is said to be a function of the $n$ independent variables $x_1$ to $x_n$. 
    We also call the $x_j$'s the function's input variables and call $w$ the function's output variable.
    \end{definition}

    \pause

    \textbf{FOCUS:} two-variable functions. Higher dimensions will be the same.
\end{frame}

\begin{frame}
    \begin{definition}
Let $f$ be a function of two variables whose domain $D$ includes points arbitrarily close to $(a,b)$. Then we say that the limit of $f(x,y)$ as $(x,y)$ approaches $(a,b)$ is $L$ and we write

$$\lim_{(x,y)\to(a,b)} f(x,y) = L$$

if for every number $\epsilon > 0$ there is a corresponding number $\delta > 0$ such that
$|f(x,y) - L| < \epsilon$
if $(x,y) \in D$ and $0 < \sqrt{(x-a)^2 + (y-b)^2} < \delta$.
    \end{definition}
\end{frame}
\end{document}

