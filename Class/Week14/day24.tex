\documentclass[aspectratio=169]{beamer}
\usetheme{Copenhagen}
%% Remove draft for real article, put twocolumn for two columns
\usetheme{metropolis}
\usepackage{multicol}
\usepackage[style=british]{csquotes}

\def\signed #1{{\leavevmode\unskip\nobreak\hfil\penalty50\hskip1em
  \hbox{}\nobreak\hfill #1%
  \parfillskip=0pt \finalhyphendemerits=0 \endgraf}}

\newsavebox\mybox
\newenvironment{aquote}[1]
  {\savebox\mybox{#1}\begin{quote}\openautoquote\hspace*{-.7ex}}
  {\unskip\closeautoquote\vspace*{1mm}\signed{\usebox\mybox}\end{quote}}

\usepackage[utf8]{inputenc}
\newtheorem*{question}{Question}
\newtheorem*{proposition}{Proposition}

\newcommand{\vectorproj}[2][]{\mathrm{proj}_{\vect{#1}}\vect{#2}}
\newcommand{\vectorcomp}[2][]{\mathrm{comp}_{\vect{#1}}\vect{#2}}
\newcommand{\vect}{\mathbf}
\newcommand{\R}{\mathbb{R}}
%% commentary bubble
\newcommand{\SV}[2][]{\sidenote[colback=green!10]{\textbf{SV\xspace #1:} #2}}

\title{ Multivariable Calculus \\ Day  24 \\ Vector Calculus: Surface integrals}

\date{Spring 2023}

\begin{document}

\maketitle

\begin{frame}
    \frametitle{Curl and divergence}
\begin{equation*}
    \mathrm{curl}\,\mathbf{F} =  \left( \frac{\partial R}{\partial y} - \frac{\partial Q}{\partial z} \right) \mathbf{i}
                       + \left( \frac{\partial P}{\partial z} - \frac{\partial R}{\partial x}  \right)  \mathbf{j}
                         + \left( \frac{\partial Q}{\partial x} - \frac{\partial P}{\partial y}  \right) \mathbf{k} \,.
\end{equation*}
\begin{equation*}
    \mathrm{div}\,\mathbf{F} =  
    \frac{\partial P}{\partial x} + \frac{\partial Q}{\partial y} + \frac{\partial R}{\partial z} \,.
\end{equation*}
    \url{https://www.youtube.com/watch?v=rB83DpBJQsE}
\end{frame}


\begin{frame}
    \frametitle{Orientation of a surface}
Given a surface \(S\), we define the orientation of it as following

\begin{enumerate}
\item
  If \(S\) has a boundary, then the \textbf{positive orientation} of the surface is that
  when one walks along the boundary of the surface with the head points in that direction, the surface is on the left.
\item
  If \(S\) does not have a boundary, then the \textbf{positive orientation} is the direction of the outward normal vector.
\end{enumerate}
Unless specified otherwise, the normal vector of a surface is conventionally
be thought of as pointing in the positive direction.
\end{frame}

\begin{frame}
    \frametitle{Parametrization}
\begin{equation*}
\mathbf{r}: D\subseteq \mathbb{R}^2 \to \mathbb{R}^3 \,.
\end{equation*}
We often write
\begin{equation*}
    \mathbf{r}(u,v) = x(u,v) \mathbf{i} + y(u,v) \mathbf{j} + z(u,v) \mathbf{k} \,.
\end{equation*}

\end{frame}


\begin{frame}
    \frametitle{Surface integral}

\begin{definition}
Let \(S\) be a surface with parametrization.
The surface integral of \(f\) over the surface \(S\) is
\begin{equation*}
    \iint_S f(x,y,z) \, dS = \lim_{m,n\to \infty} \sum_{i=1}^m \sum_{j=1}^n f(P_{ij}^*) \Delta S_{ij} \,.
\end{equation*}
\end{definition}

Similarly to the line integral,
one can show that
\begin{equation*}
   \iint_S f(x,y,z) \, dS  = \iint_D f(\mathbf{r}(u,v)) | \mathbf{r_u}\times \mathbf{r_v} | \, dA \,. 
\end{equation*}
\end{frame}

\begin{frame}
    \frametitle{Surface integral of vector fields}
\begin{definition}
If \(\mathbf{F}\) is a continuous vector field on an oriented surface \(S\) (parametrized by \(\mathbf{r}(u,v)\))
with unit normal vector \(\mathbf{n}\), then the \textbf{surface integral of \(\mathbf{F}\) over
\(S\)} is
\begin{equation*}
    \iint_S \mathbf{F}\cdot \, d\mathbf{S} = \iint_S \mathbf{F}\cdot \mathbf{n} \, dS 
    = \iint_S \mathbf{F}\cdot (\mathbf{r}_u\times \mathbf{r}_v) \, dA \,.
\end{equation*}
The integral is called the \textbf{flux of \(\mathbf{F}\)} across \(S\).
\end{definition}
\end{frame}


\begin{frame}
    \frametitle{Stokes' Theorem}
    \begin{theorem}
Let \(S\) be an oriented smooth surface that is bounded by a simple closed
smooth boundary curve \(\partial S\) with positive orientation.
Let \(\mathbf{F}\) be a vector field whose components have continuous partial
derivatives on an open region in \(\mathbb{R}^3\) that contains \(S\).
Then
\begin{equation*}
    \int_{\partial S} \mathbf{F} \cdot \, d\mathbf{r} = \iint_S \nabla \times \mathbf{F} \cdot d\mathbf{S} \,.
\end{equation*}
\end{theorem}
\url{https://www.youtube.com/watch?v=LqNqqidw2mg}
\end{frame}

\begin{frame}
    \frametitle{Divergence Theorem}
    \begin{theorem}
Let \(E\) be a simple solid region and let surface \(\partial E\) be the boundary of \(E\),
given with positive (outward) orientation.
Let \(\mathbf{F}\) be a vector field whose components have continuous partial derivatives.
Then,
\begin{equation*}
    \iint_{\partial E} \mathbf{F} \cdot d\mathbf{S} = \iiint_E \mathrm{div} \mathbf{F} \, dV \,.
\end{equation*}
\end{theorem}

\url{https://www.youtube.com/watch?v=TORt20_HjMY}

\end{frame}




\end{document}

