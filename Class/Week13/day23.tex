\documentclass[aspectratio=169]{beamer}
\usetheme{Copenhagen}
%% Remove draft for real article, put twocolumn for two columns
\usetheme{metropolis}
\usepackage{multicol}
\usepackage[style=british]{csquotes}

\def\signed #1{{\leavevmode\unskip\nobreak\hfil\penalty50\hskip1em
  \hbox{}\nobreak\hfill #1%
  \parfillskip=0pt \finalhyphendemerits=0 \endgraf}}

\newsavebox\mybox
\newenvironment{aquote}[1]
  {\savebox\mybox{#1}\begin{quote}\openautoquote\hspace*{-.7ex}}
  {\unskip\closeautoquote\vspace*{1mm}\signed{\usebox\mybox}\end{quote}}

\usepackage[utf8]{inputenc}
\newtheorem*{question}{Question}
\newtheorem*{proposition}{Proposition}

\newcommand{\vectorproj}[2][]{\mathrm{proj}_{\vect{#1}}\vect{#2}}
\newcommand{\vectorcomp}[2][]{\mathrm{comp}_{\vect{#1}}\vect{#2}}
\newcommand{\vect}{\mathbf}
\newcommand{\R}{\mathbb{R}}
%% commentary bubble
\newcommand{\SV}[2][]{\sidenote[colback=green!10]{\textbf{SV\xspace #1:} #2}}

\title{ Multivariable Calculus \\ Day  23 \\ Vector Calculus: Line integrals (cont.)}
\date{Spring 2023}

\begin{document}

\maketitle


\begin{frame}
    \frametitle{Recap worksheet}
    Evaluate
    \begin{equation*}
        \int_{C} y^2 \, dx + x\, dy \,, \qquad i = 1,2
    \end{equation*}
    \begin{enumerate}
        \item where \(C\) is the line segment from \((-5,-3) \to (0,2)\) 
        \item where \(C\) is the arc of the parabola \(x = 4-y^2\) from \((-5,-3) \to (0,2)\).
        \item repeat the above two steps with 
            \begin{equation*}
                P = x\,, \qquad Q = y \,.
            \end{equation*}
    \end{enumerate}
\end{frame}


\begin{frame}
    \frametitle{Question}
    When is a vector field $\vect{F}$ in $\R^2$ conservative?
\end{frame}


\begin{frame}
\begin{definition}
Let \(\mathbf{F}\) be a continuous vector field with domain \(D\), we say that the
line integral
\begin{equation*}
    \int_C \mathbf{F} \cdot d\mathbf{r} 
\end{equation*}
is \textbf{independent of path} if
\begin{equation*}
    \int_{C_1} \mathbf{F}\cdot d\mathbf{r} 
    =
    \int_{C_2} \mathbf{F} \cdot d\mathbf{r} 
\end{equation*}
\end{definition}
\end{frame}


\begin{frame}
    \frametitle{Independence of path and conservative vector fields}
    \begin{theorem}
    \(\int_C \mathbf{F}\cdot d\mathbf{r}\) is independent of path in \(D\) if and only if
    \(\oint_\Gamma \mathbf{F} \cdot d\mathbf{r} = 0\) for every closed path \(\Gamma \) in \(D\).
    \end{theorem}
\end{frame}

\begin{frame}
    \frametitle{Worksheet}
    Prove the above theorem.
\end{frame}

\begin{frame}
\begin{definition}
A domain \(D\) is said to be \textbf{open} if around each point, we can draw an open ball around it.
A domain \(D\) is said to be \textbf{connected} if for any two points, there is a path that connect them
together.
A domain \(D\) is said to be \textbf{simply connected} if is connected and there's no hole in it.
\end{definition}
\end{frame}


\begin{frame}
\begin{theorem}
Suppose \(\mathbf{F}\) is a vector field that is continuous on an open
connected region \(D\).
If \(\int_C \mathbf{F} \cdot d \mathbf{r}\) is independent of path in \(D\),
then \(\mathbf{F}\) is a conservative vector field on \(D\).
\end{theorem}
\end{frame}

\begin{frame}
    \frametitle{Is there an easier way?}
    \pause
    Clairaut's theorem
    If $\vect{F}$ is conservative then
    \begin{equation*}
        \frac{\partial P}{\partial y} = \frac{\partial Q}{\partial x} \,.
    \end{equation*}

    \pause

    Is the converse true?
\end{frame}

\begin{frame}
    What happens when $\vect{F}$ is not conservative?
\end{frame}

\begin{frame}
    \frametitle{Green's Theorem}
\begin{theorem}[Green's Theorem]
Let \(D\) be an open bounded simply connected domain in \(\mathbb{R}^2\),
\(\Gamma\) be the boundary of \(D\),
and \(\mathbf{F} = P\mathbf{i} + Q \mathbf{j}\) be a vector field.
If \(P\) and \(Q\) have continuous partial derivatives on an open region
that contains \(D\), then
\begin{equation*}
    \int_\Gamma \mathbf{F} \cdot d \ell  = \iint_D \left( \frac{\partial Q}{\partial x} - \frac{\partial P}{\partial y} \right) \, dA \,.
\end{equation*}
\end{theorem}
\end{frame}


\begin{frame}
\begin{theorem}
Let \(\mathbf{F} = P\mathbf{i} + Q\mathbf{j}\) be a vector field on an open simply connected
region \(D\). Suppose that
\(P\) and \(Q\) have continuous first-order partial derivatives and
\begin{equation*}
    \frac{\partial P}{\partial y} = \frac{\partial Q}{\partial x}
\end{equation*}
through out \(D\).
Then \(\mathbf{F}\) is conservative.
\end{theorem}
\end{frame}

\begin{frame}
    \frametitle{Computing the area of any region bounded by a curve}
    \url{https://www.youtube.com/watch?v=aLSx1eM27P4}
\end{frame}



\end{document}

