\documentclass{amsart}
%% Remove draft for real article, put twocolumn for two columns

\usepackage{amsmath,amssymb,amsthm}
\usepackage[utf8]{inputenc}
\usepackage{url}
\usepackage{hyperref}
\newtheorem{question}{Question}
\theoremstyle{definition}
\newtheorem{problem}{Problem}
\newtheorem{theorem}{Theorem}
\newcommand{\R}{\mathbb{R}}

\newcommand{\vectorproj}[2][]{\mathrm{proj}_{\vect{#1}}\vect{#2}}
\newcommand{\vectorcomp}[2][]{\mathrm{comp}_{\vect{#1}}\vect{#2}}
\newcommand{\vect}{\mathbf}
%% commentary bubble
\newcommand{\SV}[2][]{\sidenote[colback=green!10]{\textbf{SV\xspace #1:} #2}}

%% Title 
\title{ Multivariable Calculus \\ Day 11 Worksheet \\ Differentiability and  Directional Derivative } 
%\author[1]{Co-author}
\date{Spring 2023}

\begin{document}

\maketitle

\begin{question}
    We will deal with implicit differentiation.
    Re-call in calculus, there are situations like the following
    \begin{equation}
        \label{eq:implicit1}
        y^2 + y = e^x \,,
    \end{equation}
    and you were asked to compute $\frac{dy}{dx}$.
    Now, you will learn how to do similar things with more variables.
    \begin{itemize}
    \item Compute $\frac{dy}{dx}$ for~\eqref{eq:implicit1}.
    \item Suppose  $F(x,y,z) = 0$.
        Use the chain rule to find a formula for $\frac{\partial y}{\partial x}$ 
        (or $\frac{\partial x}{\partial z}$).
    \item Let $z = \sin(x) + e^{xy}$.
        Compute $\frac{\partial x}{\partial y}$ and $\frac{\partial z}{\partial x}$.
    \end{itemize}
\end{question}


\begin{question}
    Read section 10.5.2 of the book Active Calculus
    about the tree diagrams to represent 
    the chain rule.
    \url{https://activecalculus.org/multi/S-10-5-Chain-Rule.html}
    
    Complete Activity 10.5.3.
\end{question}

\begin{question}
    Let's think about the following theorem.
    \begin{theorem}
        Let $f:\R^n \to \R$ be a function.
        \begin{enumerate}
            \item If $f$ is differentiable at $\vect{x}_0$, then $f$ is continuous at $\vect{x}_0$.
            \item If $f$ is differentiable at $\vect{x}_0$, then $\partial_{x_i} f$ exists for every $i = 1, \dots, n$.
            \item If $\partial_{x_i} f$ exists AND is continuous for all $i = 1,\dots, n$,
                then $f$ is continuously differentiable (differentiable and the derivative
                is continuous (whatever that means)).
        \end{enumerate}
    \end{theorem}
    To have some feel for this theorem, let's try the following.
Compute directional derivative in the direction $\vect{u} \not= \vect{0}$ of the functions
\begin{itemize}
    \item $f(\vect{x}) = |\vect{x}|^2$.
    \item $$ f(\vect{x}) = 
        \begin{cases}
            \frac{x_1^2 x_2}{x_1^4 + x_2^2} & \vect{x} \not= 0 \\
            0 & \vect{x} = 0 \,.
        \end{cases}$$
    \item Does having directional derivative in ALL direction $\vect{u}$ imply 
        that the function $f$ is differentiable?
    \item Read the following article and make some sort of flow-chart / mind map 
        to cover all of the connections between differentiability and
        partial derivative. 
        \url{https://mathinsight.org/differentiable_function_discontinuous_partial_derivatives}
\end{itemize}
\end{question}


\begin{question}
    Re-call that Taylor's theorem for one dimensional a smooth function 
    $f$ is
    \begin{equation*}
        f(x) = f(a) + f'(a)(x-a) + \frac{1}{2} f''(b)(x-a)^2 \,,
    \end{equation*}
    where $b$ is a number in between $a$ and $x$.

    \begin{itemize}
        \item The first two terms of the Taylor's theorem gives us the linear 
            approximation of $f$. This is a little bit puzzling.
            Without any restriction, a linear approximation should be
            a function 
            $T(x) = mx + b$ where $T(a) = f(a)$.
            That means, any line that would coincide with $f$ at $a$ should be
            okay.

            However, you were taught that there's only one linear approximation.
            Surely,  there must be other condition(s) that help determine that one line
            you were taught.
            Discuss among friends to find out what's missing.

        \item Find the linear approximation for $f(x) = e^{x^2}$.
        
        \item Generalize the previous two questions to find appropriate conditions
            to have a good linear approximation for the function
        $f(\vect{x}) = e^{|\vect{x}|^2}$, where $\vect{x} \in \R^2$.
    \item Find the linear approximation $L(\vect{x})$ for $f(\vect{x})$.
    \end{itemize}
\end{question}

\begin{question}
    Relate what you've just done in the previous question to the definition 
    of differentiability. 
    Explain the meaning of the definition in human language.
\end{question}

\begin{question}
    Read more about differentials in Stewart Section 14.4 to get to know the
    terminology.
    This terminology is used a lot but it is just a convention with no clear
    definition.
    Most elementary calculus books (including the books we use) 
    go as far as letting $dx = \Delta x$, which is absolutely horrendous. 
\end{question}



\end{document}
