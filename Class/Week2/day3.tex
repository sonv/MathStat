\documentclass[aspectratio=169]{beamer}
\usetheme{Copenhagen}
%% Remove draft for real article, put twocolumn for two columns
\usetheme{metropolis}

\usepackage[utf8]{inputenc}
\newtheorem*{question}{Question}

\newcommand{\vectorproj}[2][]{\mathrm{proj}_{\vect{#1}}\vect{#2}}
\newcommand{\vectorcomp}[2][]{\mathrm{comp}_{\vect{#1}}\vect{#2}}
\newcommand{\vect}{\mathbf}
\newcommand{\R}{\mathbb{R}}
%% commentary bubble
\newcommand{\SV}[2][]{\sidenote[colback=green!10]{\textbf{SV\xspace #1:} #2}}

%% Title 
\title{ Multivariable Calculus \\ Day 3 \\ Some toy examples in $\R^3$} 
\institute{Fulbright University Vietnam}
%\author[1]{Co-author}
\author{Truong-Son Van}
\date{Spring 2023}

\begin{document}

\maketitle

\section{Equations for lines and planes}

\begin{frame}
    \frametitle{Equation for a line}
A line is a collection of points that is parallel to a vector and goes through a 
\begin{equation*}
    L = \{\vect{r}(t) \,|  \vect{r}(t) = \vect{r}_0 + t \vect{v}, t\in \R \}  \,,
\end{equation*}
where ${r}_0$ is the initial position and $\vect{v}$ is the direction.
The equation for $\vect{r}(t)$ is called a \textbf{vector equation for a line $L$}.
\end{frame}

\begin{frame}
    \frametitle{Equation for a line}
Let $\vect{v} = \langle v_1, v_2, v_3 \rangle$ and $\vect{r}_0 = ( x_0, y_0, z_0 )$.
The \textbf{parametric equations} of $L$ is the following system of equations

\begin{gather*}
    x = x_0 + v_1 t\,, \\
    y = y_0 + v_2 t\,, \\
    z = z_0 + v_3 t \,. 
\end{gather*}

This leads to the \textbf{symmetric equations} of $L$

\begin{equation*}
    \frac{x - x_0}{v_1} = \frac{y - y_0}{v_2} = \frac{z - z_0}{v_3} \,.
\end{equation*}
\end{frame}

\begin{frame}
    \frametitle{Equation for a line}
    Two lines are parallel if their directional vectors are parallel (scalar multiple of each other).

    Two lines that are not parallel and don't intersect each other are said to be skewed.
\end{frame}

\begin{frame}
    \frametitle{Worksheet}
    \begin{enumerate}
                \item Find parametric equations and symmetric equations of the line that passes 
    through the points $A(2,4,-4)$ and $B(3,-1,1)$.
         At what point does this line intersect the $xy$-plane?
        \item What would be an equation that describe the line segment 
            connecting two points $A$ and $B$ in $\R^3$?
        \item Determine if the following lines are skew or not
            $$L_1: x= 1 +t, y = -2 + 3t, z= 4-t$$
            $$ L_2: x = 2s, y = 3+s, z= -3 + 4s$$
    \end{enumerate}
\end{frame}


\begin{frame}
    \frametitle{Equation for plane}
A plane is a collection of points that is perpendicular to one specific direction 
\begin{equation*}
    P = \{ \vect{r} \, | \, \vect{n} \cdot (\vect{r}- \vect{r}_0 ) = 0 \} \,.
\end{equation*}
$\vect{n}$ is the perpendicular vector to the plane called the normal vector.

This is called a vector equation for the plane $P$.

In higher dimension, planes are called hyperplanes.
This is an important concept in data classification.
\end{frame}

\begin{frame}
Multiplying things out, we have the scalar equation of the plane $P$ with 
normal vector $\vect{n} = \langle n_1, n_2, n_3 \rangle$ through a point $P_0(x_0, y_0, z_0)$
\begin{equation*}
    n_1(r_1- x_0) + n_2 (r_2 - y_0) + n_3(r_3 - z_0) = 0 \,.
\end{equation*}
This is called the scalar equation for the plane $P$.

The equation of the form
\begin{equation*}
    ax + by + cz + d = 0 
\end{equation*}
is called a linear equation.
\end{frame}

\begin{frame}
    \frametitle{Worksheet}
    \begin{enumerate}
        \item Find an equation of the plane through the point $(2,4,-1)$
            with normal vector $\vect{n} = \langle 2,3,4 \rangle$.
            Sketch the plane on the coordinate system.
        \item Find the angle between the planes $x+y+z = 1$ and 
            $x -2y + 3z =1$. Find an equation for the line of intersection.
        \item Find a formula for the distance from a point $P(x_1,y_1,z_1)$
            to the plane $ax +by + cz +d = 0 $.
    \end{enumerate}
\end{frame}





\end{document}

