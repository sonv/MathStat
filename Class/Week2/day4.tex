\documentclass[aspectratio=169]{beamer}
\usetheme{Copenhagen}
%% Remove draft for real article, put twocolumn for two columns
\usetheme{metropolis}

\usepackage[utf8]{inputenc}
\newtheorem*{question}{Question}

\newcommand{\vectorproj}[2][]{\mathrm{proj}_{\vect{#1}}\vect{#2}}
\newcommand{\vectorcomp}[2][]{\mathrm{comp}_{\vect{#1}}\vect{#2}}
\newcommand{\vect}{\mathbf}
\newcommand{\R}{\mathbb{R}}
%% commentary bubble
\newcommand{\SV}[2][]{\sidenote[colback=green!10]{\textbf{SV\xspace #1:} #2}}

%% Title 
\title{ Multivariable Calculus \\ Day 4 \\ Multivariable and vector functions} 
\institute{Fulbright University Vietnam}
%\author[1]{Co-author}
\author{Truong-Son Van}
\date{Spring 2023}

\begin{document}

\maketitle


\begin{frame}
    \frametitle{From last time}
    \begin{enumerate}
        \item Find the angle between the planes $x+y+z = 1$ and 
            $x -2y + 3z =1$. Find an equation for the line of intersection.
        \item Find a formula for the distance from a point $P(x_1,y_1,z_1)$
            to the plane $ax +by + cz +d = 0 $.
    \end{enumerate}
\end{frame}

\begin{frame}
    \frametitle{Quadric surfaces}
    A quadric surface is the graph of a second-degree equation in three variables
    $x,y$ and $z$. 
    The equation that represents these surfaces is
    $$Ax^2 + By^2 + Cz^2 + Dz = E\,.$$
\end{frame}


\section{Functions of several variables}
\begin{frame}
\begin{definition}
    \frametitle{\secname}
A function of several variables is a function
$f: D \to C$ where $D \subseteq \R^m$ and $C \subseteq \R^n$.
$$f({x}) = ( f_1(x_1,\dots, x_m),\dots, f_n(x_1,\dots, x_m)  ) \,.$$
$D$ is called the domain of $f$ and $C$ is called the codomain of $f$.
\end{definition}
\end{frame}

\begin{frame}
    \frametitle{Examples}

    \begin{enumerate}
        \item $f(x,y) = x^2 - 2xy + y^2$
        \item $f(x,y,z) = \frac{1}{1 - xy^2}$
    \end{enumerate}

\end{frame}

\section{Vector functions}

\begin{frame}
    \frametitle{\secname}
\begin{definition}
A vector function (vector-valued function) is a function that has the codomain that belongs to $\R^n$ where $n\geq 2$. In other words, $f: D  \to \R^n$.
\end{definition}

\begin{example}
The following are some examples of vector functions.

\begin{itemize}
    \item  Line: $\vect{r}(t) = \vect{r}_0 + t\vect{v}$
    \item  Helix: $\vect{f}(t) = \langle \cos(t),\sin(t), t \rangle$
\end{itemize}
\end{example}
\end{frame}

\begin{frame}
    \begin{theorem}
    Let $\vect{r}: \R \to \R^n$, given by $\vect{r}(t) = \langle r_1(t), \dots , r_n(t) \rangle$.
    Then, $\vect{r}$ is said to be continuous at $t_0$ if
    \begin{equation*}
        \vect{r}(t_0) = \lim_{t\to t_0} \vect{r}(t) \,,
    \end{equation*}
    where
    \begin{equation*}
        \lim_{t\to t_0} \vect{r}(t) = \langle \lim_{t\to t_0}r_1(t) , \dots , \lim_{t\to t_0} r_n(t) \rangle \,. 
    \end{equation*}
    Furthermore, we can define the derivative of $\vect{r}$
    \begin{equation*}
        \frac{d}{dt} \vect{r}(t) = \vect{r}'(t) = \lim_{h\to 0} \frac{\vect{r}(t+h) - \vect{r}(t)}{h} 
    \end{equation*}
    if this limit exists.
    \end{theorem}
\end{frame}
\begin{frame}
    \frametitle{Worksheet}
    \begin{itemize}
        \item Draw the helix $\vect{f}(t) = \langle \cos(t),\sin(t), t \rangle$
        \item Find $\lim_{t\to \pi} \vect{f}(t)$
        \item Find $\vect{f}'(t)$
        \item Fill in the right-hand side
            \begin{enumerate}
            \item  $(\vect{u}(t) + \vect{v}(t))' = $
            \item  $(c \vect{u}(t))' = $
            \item  $(f(t) \vect{u}(t))' = $
            \item  $(\vect{u}(t) \cdot \vect{v}(t))' = $
            \item  $(\vect{u}(t) \times \vect{v}(t))' = $
            \item  $(\vect{u}(f(t)))' = $
            \end{enumerate}
    \end{itemize}
\end{frame}

\end{document}

