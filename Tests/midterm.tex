\documentclass[12pt]{article}
\usepackage{marktext} 
%% Remove draft for real article, put twocolumn for two columns
\usepackage{svmacro}
\usepackage[utf8]{inputenc}
\usepackage[style=alphabetic, backend=biber]{biblatex}
\addbibresource{bibliography.bib}

%% commentary bubble
\newcommand{\SV}[2][]{\sidenote[colback=green!10]{\textbf{SV\xspace #1:} #2}}

%% Title 
\title{ MATH 104: Multivariable Calculus}
%\author[1]{Co-author}
\author{Name: \_ \_ \_ \_ \_ \_ \_ \_ \_ \_ \_ \_ \_ \_ \_ \_ \_}
%\affil[1]{Institute}
\date{\today}

\begin{document}

\maketitle

\section*{Rules}

\begin{itemize}
    \item 5 questions, 90 minutes
    \item Closed books
    \item No calculator
    \item Show all your work. Mere numbers for solutions will not count for grades.
\end{itemize}

\section*{Scores}

\begin{enumerate}[Problem 1.]
    \item  \_\_\_/20
    \item  \_\_\_/20
    \item \_\_\_/20
    \item  \_\_\_/20
    \item  \_\_\_/20
\end{enumerate}
Total \_\_\_\_\_\_\_\_/100


\newpage
\section*{Questions}

\begin{problem}Let $f:\R^2 \to \R$.
    \begin{enumerate}[a.]
        \item  What does it mean for $f$ to be 
            differentiable at $(a,b)$?
        \item What does it mean for $f$ to have a directional derivative
            in the direction of $\textbf{u}$? What's a notation for this notion?
        \item Write directional derivative of function $f$ in the direction $\textbf{u}$ in terms of partial derivative/gradient of $f$.
        \item What is the dimension of $\grad f$? 
            What about $D_{\textbf{u}}f$? 
            What about $\lap f$?
    \end{enumerate}
\end{problem}


\newpage
\begin{problem}
    \begin{enumerate}[a.]
        \item State Clairaut's theorem.
        \item An example that was introduced in class but we didn't have 
            time to cover in details. Hopefully you've done that at home.
            If you have, the following will be easy enough.
            Show that $f_{xy}(0,0) \not= f_{yx}(0,0)$ for the following function
            \begin{equation*}
                f(x,y) = 
                \begin{dcases}
                    \frac{xy(x^2 - y^2)}{x^2 + y^2} \,, & (x,y) \not= (0,0) \\
                    0 \,, & (x,y) = (0,0) \,.
                \end{dcases}
            \end{equation*}
        \item Why doesn't the above function follow Clairaut's theorem?
    \end{enumerate}
\end{problem}


\newpage
\begin{problem}
\begin{enumerate}
    \item Given a function $f:\R^n \to \R$, where $n\geq 2$.
        What are the properties of $\grad f$? (List everything that you can think of)
    \item Given any surface $F(x,y,z) = C$, where $F$ is differentiable everywhere, how do you know there's only one tangent plane at a given point?
        (Give the best answer you can)
    \item Find the tangent plane to the surface
        \begin{equation*}
            x^2 + 2xy - y^2 + z^2 =7 
        \end{equation*}
        at point
        \begin{equation*}
            P_0(1,-1,3) \,.
        \end{equation*}
\end{enumerate}
\end{problem}


\newpage
\begin{problem}
    \begin{enumerate}
        \item Find the tangent vector, normal vector and curvature of the following curve
    \begin{equation*}
        \textbf{r}(t) = \langle \cos t + t \sin t, \sin t - t\cos t, 3 \rangle \,.
    \end{equation*}
    \item What's the meaning of the curvature of a curve at a point on the curve?
    \end{enumerate}

\end{problem}

\newpage

\begin{problem}
    \begin{enumerate}
        \item State the second derivative test when you want to optimize a differentiable function $f:\R^2 \to \R$.
        \item Find all the local maxima, local minima, and saddle points (neither min nor max) of the function
    \begin{equation*}
        f(x,y) = \ln(x+y) + x^2 - y \,.
    \end{equation*}
    \end{enumerate}
\end{problem}
%\bibliography{refs}


\end{document}
