\documentclass[12pt]{amsart}
\usepackage{marktext} 
%% Remove draft for real article, put twocolumn for two columns
\usepackage{svmacro}
\usepackage[utf8]{inputenc}
\usepackage[style=alphabetic, backend=biber]{biblatex}
\addbibresource{bibliography.bib}

%% commentary bubble
\newcommand{\SV}[2][]{\sidenote[colback=green!10]{\textbf{SV\xspace #1:} #2}}
\newcommand{\vectorproj}[2][]{\mathrm{proj}_{\vect{#1}}\vect{#2}}
\newcommand{\vectorcomp}[2][]{\mathrm{comp}_{\vect{#1}}\vect{#2}}
\newcommand{\vect}{\mathbf}

%% Title 
\title{ On osculating circle and curvature in 2D }
\author{Truong-Son Van}
%\affil[1]{Institute}
\address{Fulbright University Vietnam}
\email{son.van@fulbright.edu.vn}

\date{\today}

\begin{document}


\maketitle


\section{Introduction}
In this expository note, we derive the relationship between curvature and the radius of the osculating circle.
In particular, let $p$ be a point on a twice differentiable curve, $R$ be
the radius of the osculating circle at that point and $\kappa$ is the curvature of the curve 
at that point. Then,
\begin{equation*}
    R = \frac{1}{\kappa} \,.
\end{equation*}
There is an article by Fuchs~\cite{Fuchs2013} about this for those who are interested in the geometry of curves.

From Wikipedia, the definition of an osculating circle is the following.

\begin{definition}
The osculating circle of a sufficiently smooth plane curve at a given point $p$
on the curve has been traditionally defined as the circle passing through $p$
and a pair of additional points on the curve infinitesimally close to $p$.
\end{definition}

\section{Derivation}

Let $\vect{r}:[a,b]\to \R^2$  be a space curve that is twice differentiable.
For a point $A=\vect{r}(t)$ on the curve, where $t\in (a,b)$, we have two other points,
 $B=\vect{r}(t-\epsilon)$ and $C=\vect{r}(t+\epsilon)$ given that $\epsilon$ is small enough.

 Furthermore, given any 3 points in $\R^2$, there exists a unique circle that go through
 all 3 of them. 
 To determine this circle, we build two center lines\footnote{A center line 
 of a segment is the line that is perpendicular at the center of that segment}
     that divide the segments $AB$ and $AC$.
     The center of the circle that passes through $A,B,C$  would be the intersection of these two lines.

     Mathematically, denote $c^\epsilon(t)$ to be the center of the circle passing through $A,B,C$.
     We have the following pair of equations
     \begin{align*}
         \left(c^\epsilon(t) - \frac{\vect{r}(t) + \vect{r}(t-\epsilon)}{2} \right)\cdot\left( \vect{r}(t) - \vect{r}(t-\epsilon) \right) = 0  \\
         \left(c^\epsilon(t) - \frac{\vect{r}(t) + \vect{r}(t+\epsilon)}{2} \right)\cdot\left( \vect{r}(t) - \vect{r}(t+\epsilon) \right) = 0  
     \end{align*}

     Assume that there is a limit $c^\epsilon(t) \to c(t)$ as $\epsilon \to 0$ ($c(t)$ will be the center of the osculating circle).
     Dividing everything by $\epsilon$ and letting $\epsilon\to 0$, we have
     \begin{equation*}
         \left( \vect{c}(t) - \vect{r}(t) \right) \vect{r}'(t) = 0 \,.
     \end{equation*}

     This implies that the center of the osculating circle lies on the normal direction of 
     the curve. 
     The nice thing about 2D is that, we can produce a formula for the normal direction of the
     curve at $\vect{r}(t)$ explicitly.
     Let $\vect{n}(t) = \langle -r_2'(t) , r_1'(t) \rangle$.
     Then, $\vect{n}(t) \cdot \vect{r}'(t) = 0$.

     For $s\not= t$, the two normal lines at $\vect{r}(t)$ and $\vect{r}(s)$ will intersect
     at a point $P(s)$. Furthermore,
     \begin{equation*}
     \lim_{s\to t} P(s) = \vect{c}(t) \,.
     \end{equation*}

     Let's now find the formula for $P(s)$.
     In particular, let
     \begin{equation*}
         P(s) = \vect{r}(t) + \eta \vect{n}(t) = \vect{r}(t) + \lambda \vect{n}(s)  \,.
     \end{equation*}
     This means,
     \begin{equation*}
         \vect{r}(t) + \eta \langle -r_2'(t) , r_1'(t) \rangle = \vect{r}(s) + \lambda \langle -r_2'(s), r_1'(s) \rangle \,.
     \end{equation*}

     Writing this in the form of system of equations, we have
     \begin{equation*}
         \begin{dcases}
         r_1(t) - \eta r_2'(t) &= r_1(s) - \lambda r_2'(s) \\
         r_2(t) + \eta r_1'(t) &= r_2(s) + \lambda r_1'(s) 
         \end{dcases}
     \end{equation*}
     Multiplying the first equation by $r_1'(s)$ and the second equation by $r_2'(s)$, we have
     \begin{equation*}
         \begin{dcases}
             (r_1(t) - \eta r_2'(t))r_1'(s) &= (r_1(s) - \lambda r_2'(s))r_1'(s) \\
             (r_2(t) + \eta r_1'(t))r_2'(s) &= (r_2(s) + \lambda r_1'(s))r_2'(s) 
         \end{dcases}
     \end{equation*}
     Adding both equations, we have
     \begin{equation*}
         (r_1(t) + r_2(t)) + \eta (-r_2'(t) r_1'(s) + r_1'(t) r_2'(s)) = r_1(s)r_1'(s) + r_2(s)r_2'(s) \,.
     \end{equation*}
     Therefore,
     \begin{align*}
         \eta &= \frac{ r_1(s)r_1'(s) + r_2(s)r_2'(s) - r_1(t)r_1'(s) - r_2(t)r_2'(s)}{-r_2'(t) r_1'(s) + r_1'(t) r_2'(s)}\\
              &= \frac{ r_1'(s) \frac{r_1(s) - r_1(t)}{s-t} + r_2'(s) \frac{r_2(s) - r_2(t)}{s-t}}{ r_1'(s)r_2'(s) - r_2'(t) r_1'(s) + r_1'(t) r_2'(s) - r_1'(s) r_2'(s)} \\
              &= \frac{ r_1'(s) \frac{r_1(s) - r_1(t)}{s-t} + r_2'(s) \frac{r_2(s) - r_2(t)}{s-t}}{ r_1'(s)\frac{r_2'(s) - r_2'(t)}{s-t}  + \frac{r_1'(t)  - r_1'(s)}{s-t} r_2'(s)}  \,.
     \end{align*}
     Letting $s\to t$, we have
     \begin{equation*}
         \lim_{s\to t} \eta = \frac{(r_1'(t))^2  + (r_2'(t))^2}{ r_1'(t) r_2''(t) - r_1''(t)r_2'(t)} \,.
     \end{equation*}
     Therfore,
     \begin{align*}
         \lim_{s\to t} P(s) &= \vect{r}(t) + \frac{(r_1'(t))^2  + (r_2'(t))^2}{ r_1'(t) r_2''(t) - r_1''(t)r_2'(t)} \langle -r_2'(t), r_1'(t) \rangle \\
                            &= \vect{r}(t) + \frac{ | \vect{r}'(t) |^{3/2} }{ r_1'(t) r_2''(t) - r_1''(t)r_2'(t)} \vect{u}(t) \,,
     \end{align*}
     where $ \vect{u}(t)$ is the unit normal vector at $\vect{r}(t)$.
     It is now easy to verify that
     \begin{equation*}
         \frac{\abs{ \vect{r}'(t)}}{\abs{\vect{T}'(t)}}= \frac{ | \vect{r}'(t) |^{3/2} }{ r_1'(t) r_2''(t) - r_1''(t)r_2'(t)} 
     \end{equation*}
     as it's a routine algebraic manipulation.

    Therefore, the radius of the osculating circle is
    \begin{equation*}
        R = \frac{\abs{ \vect{r}'(t)}}{\abs{\vect{T}'(t)}}\,.
    \end{equation*}

    From Stewart~\cite{Stewart2015}, the curvature of the curve is
    \begin{equation*}
        \kappa = \frac{\abs{\vect{T}'(t)}}{\abs{ \vect{r}'(t)}} = \frac{1}{R} \,.
    \end{equation*}
     
    






\printbibliography 
%\bibliography{refs}
%\bibliographystyle{halpha-abbrv}


\end{document}
